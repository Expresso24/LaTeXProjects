\documentclass{article}




\usepackage[spanish]{babel}
\usepackage[T1]{fontenc}

% Paquetes útiles
\usepackage[utf8]{inputenc}  % Codificación UTF-8
\usepackage{amsmath}         % Soporte para fórmulas matemáticas
\usepackage{graphicx}        % Para incluir imágenes
\usepackage{hyperref}        % Para enlaces
\usepackage{geometry}        % Configurar márgenes
\geometry{a4paper, margin=2 cm}



%Huge es para grande el titulo y bfseries para negritas

\title{\Huge \bfseries BubbleSort}
\author{Ernesto Soria}
\date{\today} % Fecha automática

\begin{document}
	
	\maketitle % Genera título, autor y fecha
	
	
	
	
	\begin{figure}[h] % h = aquí, t = arriba, b = abajo, p = página aparte
		\centering
		\includegraphics[width=1\textwidth]{cetilogo.png} % Ajusta el ancho de la imagen
		\label{fig:ceti} % Permite referenciar la imagen con \ref{fig:ejemplo}
	\end{figure}
	
	\begin{center}
		{\LARGE \textbf{Centro de Enseñanza Técnica Industrial}} \\[2.5mm]
		{\LARGE \textbf{Plantel Colomos}}\\
	\end{center}
	
	\vspace{5mm} % Espaciado opcional
	
	\begin{flushleft}
		{\large \textbf{Nombre:} Ernesto David Soria Ramos} \\[2mm]
		{\large \textbf{Número de matrícula:} 23310003} \\[2mm]
		{\large \textbf{Grado y grupo:} 4-N}\\[2mm]
		{\large \textbf{Materia:} Estructura de datos y algoritmia }\\[2mm]
	\end{flushleft}
	
	\newpage
	
	


		\section{Introducción}
		El algoritmo de ordenamiento por burbuja (Bubble Sort) es un método simple para ordenar un arreglo. Compara pares de elementos adyacentes e intercambia los que están en el orden incorrecto. Este proceso se repite hasta que el arreglo está ordenado.
		
		\section{Análisis del Código}
		El código en Java sigue estos pasos:
		\begin{enumerate}
			\item Se define un arreglo desordenado con valores de ejemplo.
			\item Se implementa la función bubbleSort, que:
			\begin{itemize}
				\item Recorre el arreglo varias veces.
				\item Compara elementos adyacentes e intercambia si es necesario.
				\item En cada iteración, el elemento más grande se mueve a su posición correcta.
			\end{itemize}
			\item Se imprime el arreglo ordenado al final.
		\end{enumerate}
		
		\section{Complejidad del Algoritmo}
		\subsection*{Peor caso: \(O(n^2)\)}
		Si el arreglo está en orden inverso, se necesitan muchas comparaciones e intercambios:
		\begin{equation}
			O(n^2)
		\end{equation}
		
		\subsection*{Mejor caso: \(\Omega(n)\)}
		Si el arreglo ya está ordenado, solo se realiza una pasada sin intercambios:
		\begin{equation}
			\Omega(n)
		\end{equation}
		
		\subsection*{Caso promedio: \(\Theta(n^2)\)}
		En la mayoría de los casos, se requieren comparaciones e intercambios múltiples:
		\begin{equation}
			\Theta(n^2)
		\end{equation}
		
		\section{Desventajas del Algoritmo}
		\begin{itemize}
			\item Es ineficiente para arreglos grandes.
			\item Realiza muchos intercambios innecesarios.
			\item Existen algoritmos más rápidos como QuickSort \(O(n \log n)\) o MergeSort \(O(n \log n)\).
		\end{itemize}
		
		\section{Conclusión}
		El algoritmo Bubble Sort es fácil de entender pero poco eficiente. Su complejidad \(O(n^2)\) lo hace poco recomendable para datos grandes, aunque sigue siendo útil con fines educativos.
		

	
	
	
	
	
	
	
\end{document}
